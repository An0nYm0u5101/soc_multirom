\documentclass[12pt, a4paper, oneside]{article} 

\usepackage{czech} % nastavení češtiny
\usepackage[center]{caption} 
\usepackage[utf8]{inputenc}
\usepackage{wrapfig} % nastavení obtékání textu
\usepackage{graphicx,amsmath} % nastavení grafiky, matematiky
\usepackage{subfig} % více obrázků vedle sebe 
\usepackage{float}
\usepackage{amsmath}
\usepackage{amssymb}
\usepackage{bbding}
\usepackage{enumitem}
\usepackage{breakurl}
\usepackage{pdflscape}
%\usepackage{indentfirst}

\usepackage{tocloft} %přidá tečky do obsahu ke kapitolám /sekcím 
\renewcommand{\cftsecdotsep}{\cftdotsep}

\usepackage[bookmarksopen,colorlinks,plainpages=false,linkcolor=black,urlcolor=blue,citecolor=black,filecolor=black,menucolor=black,unicode=true]{hyperref}

\urlstyle{rm}
%bookmarksopen -- open up bookmark tree 
%colorlinks -- zbarví odkazy (implicitně orámovaný nezbarvený text)
%urlcolor -- barva odkazů (implicitně magenta) 
%linkcolor=black -- barva odkazů v obsahu (implicitně red)


\usepackage{listings}
\usepackage{color}
%\usepackage{minted}
\definecolor{lightgray}{RGB}{240,240,240}
\definecolor{darkgray}{rgb}{.4,.4,.4}
\definecolor{purple}{rgb}{0.65, 0.12, 0.82}
\definecolor{darkgreen}{RGB}{0,150,0}

%\usemintedstyle{perldoc}
%\newminted{js}{linenos=true, bgcolor=lightgray}

\lstdefinelanguage{JavaScript}{
  keywords={typeof, new, true, false, catch, function, return, null, catch, switch, var, if, in, while, do, else, case, break, for},
  keywordstyle=\color{blue}\bfseries,
  ndkeywords={class, export, boolean, throw, implements, import, this},
  ndkeywordstyle=\color{blue}\bfseries,
  identifierstyle=\color{black},
  sensitive=zr,
  comment=[l]{//},
  morecomment=[s]{/*}{*/},
  commentstyle=\color{darkgreen}\ttfamily,
  stringstyle=\color{red}\ttfamily,
  morestring=[b]',
  morestring=[b]"
}

\lstset{
   backgroundcolor=\color{lightgray},
   extendedchars=true,
   basicstyle=\footnotesize\ttfamily,
   showstringspaces=false,
   showspaces=false,
   numbers=left,
   numberstyle=\footnotesize,
   numbersep=9pt,
   tabsize=2,
   breaklines=true,
   showtabs=false,
   aboveskip=5mm,
   belowskip=7mm,
   captionpos=b
}

\renewcommand{\listingscaption}{Příklad}
\renewcommand{\listoflistingscaption}{Příklady}

% \usepackage{parskip} -- zapne americké odstavce v celé práci

\addtolength{\textwidth}{-2mm} 
\addtolength{\hoffset}{4mm}  % posun textu kvůli kroužkové vazbě  

\setlength{\intextsep}{5mm} % nastavení mezery okolo obrázků

% nastavení příkazu >\figcaption pro popis čehokoli, jako by to byly obrázky 
\makeatletter   
\newcommand\figcaption{\def\@captype{figure}\caption}
\makeatother

\renewcommand\refname{Literatura} 
%\def\bibname{PŘÍLOHA D: Reference}
%\renewcommand\bibname{PŘÍLOHA D: Reference}
% přejmenuje anglický název Reference na české Literatura


%\makeindex % příprava pro výrobu indexu (jestli ho chcete)

%%    VLNKA <fileinput>  KkSsVvZzOoUuAaIi        
% Defaultni  koncovka pro <fileinput> je  ".tex"
%FIXME: haze error
%\cstieon % Vypne chovani vlnky jako tvrde mezery v matematickem rezimu

%%%%%%%%%%%%%%%%%%%%%%%%%%%%%%%%%%%%%%%%%%%%%%%%%%%%%%%%%%%%%%%
%V PROSTŘEDÍ ROVNIC SE NESMÍ VYSKYTOVAT PRÁZDNÝ ŘÁDEK
%
%PROGRAMY VLNKA A CSINDEX SE MUSÍ SPUSTIT SAMOSTATNĚ
%%%%%%%%%%%%%%%%%%%%%%%%%%%%%%%%%%%%%%%%%%%%%%%%%%%%%%%%%%%%%%%

% definice příkazů 
\newcommand{\D}{\medskip \noindent} % nový odstavec v "americkém" formátování 
\newcommand{\B}{\textbf} %tučné písmo
\newcommand{\A}{\mathbf} %tučné písmo v matematickém režimu
\newcommand{\TO}{\ensuremath{\boldsymbol\Omega}} % tučný znak velké omega -- pro ohmy
\newcommand{\I}{\index}  % vytváří položku indexu (asi nepoužijete)
\newcommand{\Deg}[1][]{\ensuremath{{#1}^\circ}} % vysází značku stupně Celsia
\newcommand{\Def}{\footnotesize Definice: \normalsize}
\newcommand{\Pos}{\footnotesize Experiment: \normalsize}
\newcommand{\Odv}{\footnotesize Odvození: \normalsize}
\newcommand{\Vym}{\footnotesize Vymezení pojmu: \normalsize}
\newcommand{\Ob}{obrázek }
\newcommand{\It}{\textit}  % kurzíva
\newcommand{\M}{\mathrm}   % v prostředí rovnic nastaví normální písmo (místo kurzívy ) 
\newcommand{\F}{\footnotesize} % zmenšená velikost písma
\newcommand{\N}{\normalsize} % normální velikost písma
%\newcommand{\U}{\underline}  % podtržené písmo
\newcommand{\e}{\ensuremath} 
\newcommand{\Has}{\textcolor{green}{\CheckmarkBold}}
\newcommand{\NoHas}{\textcolor{red}{\XSolidBrush}}
% další příkaz se aplikuje, pouze, když jste v matematickém režimu

%\hyphenation{Pusť-me pla-tí hod-no-ty do-sa-dí-me za-da-né dal-ším}
% dělení slov, kdyby implicitní nevyhovovalo

\linespread{1.3} 
% řádkování 1,5x  
% použijete podle situace  

\unitlength=1mm % nastavení volby jednotek 

% konec hlavičky
%%%%%%%%%%%%%%%%%%%%%%%%%%%%%%%%%%%%%%%%%%%%%%%%%%%%%%%%%%%%%%%%%%%
%%%%%%%%%%%%%%%%%%%%%%%%%%%%%%%%%%%%%%%%%%%%%%%%%%%%%%%%%%%%%%%%%%%

\begin{document} % začátek textové části 

% titulní strana
\pagestyle{empty} % vynechá číslování
 
\voffset = -20mm % posun začátku textu výš
\enlargethispage{60mm} % zvětší oblast tisku pro tuto stránku   

\begin{center}
 
\Large \B{STŘEDOŠKOLSKÁ ODBORNÁ ČINNOST}

\vspace{60mm}

\Huge %\LARGE
\B{MultiROM} \\
\LARGE
\B{Nástroj pro instalaci více operačních systému na mobilní zařízení}
% na titulní straně může být stručnější, pokud je to potřeba  

\Large

\vspace{90mm}


\B{Vojtěch Boček} \\

\vspace{40mm}

\B{Brno 2014}


\end{center}

\newpage % konec titulní strany 
%%%%%%%%%%%%%%%%%%%%%%%%%%%%%%%%%%%%%%%%%%%%%%%%%%%%%%%%%%%%%%%%%%%%%%%%%%%

% vnitřní titulní strana
\voffset = -20mm % posun začátku textu výš
\enlargethispage{60mm} % zvětší oblast tisku pro tuto stránku   

\begin{center}

\Large \B{STŘEDOŠKOLSKÁ ODBORNÁ ČINNOST}  \\
\vspace{10mm}
 \normalsize 
\B{Obor SOČ: 18. Informatika}% číslo a název -- vyplníme spolu 

\vspace{45mm}

\Huge
\B{MultiROM} \\
\LARGE
\B{Nástroj pro instalaci více operačních systému na mobilní zařízení}
\end{center}
\large

\vspace{50mm}


\begin{tabbing}
\hspace{10mm} \= \hspace{30mm}  \=   \kill % nastavení zarážek 
  \> \B{Autor:}  \> \B{Vojtěch Boček}        \\[8mm] 
  \> \B{Škola:}   \> \B{SPŠ a~VOŠ technická, }     \\
  \>              \> \B{Sokolská 1, 602 00 Brno}    \\[8mm]
\end{tabbing}

\vspace{20mm}

\begin{center}
\B{Brno 2014}

\end{center}
\normalsize
%%%%%%%%%%%%%%%%%%%%%%%%%%%%%%%%%%%%%%%%%%%%%%%%%%%%%%%%%%%%%%%%%%%%%%%%%%%
\newpage  % Prohlášení o autorství  
\voffset = 0mm % posun začátku textu zpět

~ % musí to tu být, aby fungovala svislá mezera

\vspace{10mm}

\section*{Prohlášení}

Prohlašuji, že jsem svou práci vypracoval samostatně, použil jsem pouze podklady (literaturu, SW atd.) citované v~práci a~uvedené v~přiloženém seznamu a~postup při zpracování práce je v~souladu se zákonem č. 121/2000 Sb., o~právu autorském, o~právech souvisejících s~právem autorským a~o~změně některých zákonů (autorský zákon) v~platném znění. 
 
\vspace{20mm} 
 
\noindent V~Brně  dne: 6.3.2014 \hspace{50mm}                 podpis:   
 

%%%%%%%%%%%%%%%%%%%%%%%%%%%%%%%%%%%%%%%%%%%%%%%%%%%%%%%%%%%%%%%%%%%%%%%%%%%
\newpage   % Poděkování -- nepovinné 

~ % musí to tu být, aby fungovala svislá mezera
\vspace{100mm}

\section*{Poděkování}
Děkuji Jakubu Streitovi za rady, obětavou pomoc, velkou trpělivost a~podnětné připomínky poskytované během práce na tomto projektu, Martinu Vejnárovi za informace o~programátoru Shupito, panu profesorovi Mgr. Miroslavu Burdovi za velkou pomoc s~prací a~v~neposlední řadě Bc. Martinu Foučkovi za rady a~pomoc při práci s~Qt Frameworkem. Dále děkuji organizaci DDM Junior za poskytnutí podpory.
 

%%%%%%%%%%%%%%%%%%%%%%%%%%%%%%%%%%%%%%%%%%%%%%%%%%%%%%%%%%%%%%%%%%%%%%%%%%%
\newpage   % Anotace 
~ % musí to tu být, aby fungovala svislá mezera
\vspace{-20mm}

\section*{Anotace}

%MultiROM is one-of-a-kind multi-boot mod for Nexus 7. It can boot any Android ROM as well as other systems like Ubuntu Touch, Plasma Active, Bohdi Linux or WebOS port.Besides booting from device's internal memory, MultiROM can boot from USB drive connected to the device via OTG cable. The main part of MultiROM is a boot manager, which appears every time your device starts and lets you choose ROM to boot. You can see how it looks on the left image below and in gallery. ROMs are installed and managed via modified TWRP recovery. You can use standard ZIP files to install secondary Android ROMs, daily prebuilt image files to install Ubuntu Touch and MultiROM even has its own installer system, which can be used to ship other Linux-based systems.

Tato práce popisuje modifikaci pro mobilní zařízení s operačním systémem Google Android, která umožňuje instalaci více operačních systémů vedle sebe, podobně jako na PC. Může to být pouze několik různých verzí OS Google Android, ale i úplně jiné systémy - například právě se vyvíjejicí Ubuntu Touch a další.

Možnost provozovat více operačních systému na jedno zařízení je užitečná spíše pro pokročilé uživatel, podobně jako na stolních počítačích, nicméně počet aktivních uživatelů dokazuje, že zájem o tuto modifikaci rozhodně existuje.


%Tato práce popisuje komplexní sadu nástrojů pro vývoj a~ovládání libovolného zařízení schopného komunikovat po sériové lince nebo TCP socketu.

%Protože se nejedná o~jednoduchou aplikaci ani o~jednostranně zaměřenou sadu nástrojů, protože se navíc celá sada průběžně rozrůstá a~protože je rozsah a~záběr použití všech funkcí a~modulů této sady příliš velký, nelze ji stručně popsat v~omezeném prostoru anotace.

%Pro získání ucelenější představy o~tom, co vše lze pomocí sady Lorris dosáhnout, prosím nahlédněte do \hyperref[uvod]{úvodu práce}.

%Aplikace zastřešuje několik modulů. Každý modul je určen pro specifickou činnost -- parsování a zobrazování dat, programování mikročipů atd.

%Hlavním přínosem tohoto softwarového balíku je, že urychluje, zpřehledňuje a~hlavně výrazně zjednodušuje vývoj a~testování aplikací pro mikročipy, typicky programování a~řízení různých druhů robotů.

%\D \B{Klíčová slova:} analýza binárních dat, programování a~řízení robotů, vývoj pro mikrokontroléry, programování mikročipů

\addtolength{\textheight}{30mm} % prodlouží následující stránku

%%%%%%%%%%%%%%%%%%%%%%%%%%%%%%%%%%%%%%%%%%%%%%%%%%%%%%%%%%%%%%%%%%%%%%%%%%%
\newpage
\pagestyle{plain}

\setlength{\voffset}{-20mm} % posune text/obrázek na této stránce nahoru
\setcounter{page}{1}  % nastaví čítač stránek znovu od jedné

\tableofcontents  % vysází obsah

\addtolength{\textheight}{-30mm} % zkrátí následující stránku
%%%%%%%%%%%%%%%%%%%%%%%%%%%%%%%%%%%%%%%%%%%%%%%%%%%%%%%%%%%%%%%%%%%%%%%%%%%
\newpage
\setlength{\voffset}{0mm} % posune text/obrázek na této stránce, kam patří
%\pagestyle{headings} % znovu zapne číslování
\pagestyle{plain}

\section*{Úvod}
\addcontentsline{toc}{section}{Úvod} % přidá položku úvod do obsahu
\label{uvod}



\newpage
\section*{Závěr}
\addcontentsline{toc}{section}{Závěr}



%%%%%%%%%%%%%%%%%%%%%%%%%%%%%%%%%%%%%%%%%%%%%%%%%%%%%%%%%%%%%%%%%%%%%%%%%%%
\newpage
\section*{PŘÍLOHA A:}


\newpage
 \section*{PŘÍLOHA E:}
 \begin{thebibliography}{99}
\addcontentsline{toc}{section}{PŘÍLOHA E: Reference}
 %% 99 znamená, že maximální délka čísla literatury jsou dva znaky
% seznam samozřejmě změníte podle svého, tohle je pouze ukázka formátování

    \bibitem{serialchart} \It{SerialChart} -- Analyse and chart serial data from RS-232 COM ports \\
    \url{http://code.google.com/p/serialchart/}\\
    (Stav ke dni 25.\,2.\,2013)


\end{thebibliography}

\newpage
\section*{PŘÍLOHA G:}
~
\addcontentsline{toc}{section}{PŘÍLOHA G: Seznam obrázků}
\listoffigures   % seznam obrázkù 

\end{document}
